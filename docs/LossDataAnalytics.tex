\documentclass[]{book}
\usepackage{lmodern}
\usepackage{amssymb,amsmath}
\usepackage{ifxetex,ifluatex}
\usepackage{fixltx2e} % provides \textsubscript
\ifnum 0\ifxetex 1\fi\ifluatex 1\fi=0 % if pdftex
  \usepackage[T1]{fontenc}
  \usepackage[utf8]{inputenc}
\else % if luatex or xelatex
  \ifxetex
    \usepackage{mathspec}
  \else
    \usepackage{fontspec}
  \fi
  \defaultfontfeatures{Ligatures=TeX,Scale=MatchLowercase}
\fi
% use upquote if available, for straight quotes in verbatim environments
\IfFileExists{upquote.sty}{\usepackage{upquote}}{}
% use microtype if available
\IfFileExists{microtype.sty}{%
\usepackage{microtype}
\UseMicrotypeSet[protrusion]{basicmath} % disable protrusion for tt fonts
}{}
\usepackage[margin=1in]{geometry}
\usepackage{hyperref}
\hypersetup{unicode=true,
            pdftitle={Loss Data Analytics},
            pdfauthor={An open text authored by the Actuarial Community},
            pdfborder={0 0 0},
            breaklinks=true}
\urlstyle{same}  % don't use monospace font for urls
\usepackage{natbib}
\bibliographystyle{econPeriod}
\usepackage{longtable,booktabs}
\usepackage{graphicx,grffile}
\makeatletter
\def\maxwidth{\ifdim\Gin@nat@width>\linewidth\linewidth\else\Gin@nat@width\fi}
\def\maxheight{\ifdim\Gin@nat@height>\textheight\textheight\else\Gin@nat@height\fi}
\makeatother
% Scale images if necessary, so that they will not overflow the page
% margins by default, and it is still possible to overwrite the defaults
% using explicit options in \includegraphics[width, height, ...]{}
\setkeys{Gin}{width=\maxwidth,height=\maxheight,keepaspectratio}
\IfFileExists{parskip.sty}{%
\usepackage{parskip}
}{% else
\setlength{\parindent}{0pt}
\setlength{\parskip}{6pt plus 2pt minus 1pt}
}
\setlength{\emergencystretch}{3em}  % prevent overfull lines
\providecommand{\tightlist}{%
  \setlength{\itemsep}{0pt}\setlength{\parskip}{0pt}}
\setcounter{secnumdepth}{5}
% Redefines (sub)paragraphs to behave more like sections
\ifx\paragraph\undefined\else
\let\oldparagraph\paragraph
\renewcommand{\paragraph}[1]{\oldparagraph{#1}\mbox{}}
\fi
\ifx\subparagraph\undefined\else
\let\oldsubparagraph\subparagraph
\renewcommand{\subparagraph}[1]{\oldsubparagraph{#1}\mbox{}}
\fi

%%% Use protect on footnotes to avoid problems with footnotes in titles
\let\rmarkdownfootnote\footnote%
\def\footnote{\protect\rmarkdownfootnote}

%%% Change title format to be more compact
\usepackage{titling}

% Create subtitle command for use in maketitle
\newcommand{\subtitle}[1]{
  \posttitle{
    \begin{center}\large#1\end{center}
    }
}

\setlength{\droptitle}{-2em}
  \title{Loss Data Analytics}
  \pretitle{\vspace{\droptitle}\centering\huge}
  \posttitle{\par}
  \author{An open text authored by the Actuarial Community}
  \preauthor{\centering\large\emph}
  \postauthor{\par}
  \date{}
  \predate{}\postdate{}

\usepackage{booktabs}
\setcounter{secnumdepth}{2}

\usepackage{amsthm}
\newtheorem{theorem}{Theorem}[chapter]
\newtheorem{lemma}{Lemma}[chapter]
\theoremstyle{definition}
\newtheorem{definition}{Definition}[chapter]
\newtheorem{corollary}{Corollary}[chapter]
\newtheorem{proposition}{Proposition}[chapter]
\theoremstyle{definition}
\newtheorem{example}{Example}[chapter]
\theoremstyle{definition}
\newtheorem{exercise}{Exercise}[chapter]
\theoremstyle{remark}
\newtheorem*{remark}{Remark}
\newtheorem*{solution}{Solution}
\begin{document}
\maketitle

{
\setcounter{tocdepth}{2}
\tableofcontents
}
\chapter*{Preface}\label{preface}
\addcontentsline{toc}{chapter}{Preface}

\emph{Date: 01 October 2018}

\subsubsection*{Book Description}\label{book-description}
\addcontentsline{toc}{subsubsection}{Book Description}

\textbf{Loss Data Analytics} is an interactive, online, freely available
text.

\begin{itemize}
\tightlist
\item
  The online version contains many interactive objects (quizzes,
  computer demonstrations, interactive graphs, video, and the like) to
  promote \emph{deeper learning}.
\item
  A subset of the book is available for \emph{offline reading} in pdf
  and EPUB formats.
\item
  The online text will be available in multiple languages to promote
  access to a \emph{worldwide audience}.
\end{itemize}

\subsubsection*{What will success look
like?}\label{what-will-success-look-like}
\addcontentsline{toc}{subsubsection}{What will success look like?}

The online text will be freely available to a worldwide audience. The
online version will contain many interactive objects (quizzes, computer
demonstrations, interactive graphs, video, and the like) to promote
deeper learning. Moreover, a subset of the book will be available in pdf
format for low-cost printing. The online text will be available in
multiple languages to promote access to a worldwide audience.

\subsubsection*{How will the text be
used?}\label{how-will-the-text-be-used}
\addcontentsline{toc}{subsubsection}{How will the text be used?}

This book will be useful in actuarial curricula worldwide. It will cover
the loss data learning objectives of the major actuarial organizations.
Thus, it will be suitable for classroom use at universities as well as
for use by independent learners seeking to pass professional actuarial
examinations. Moreover, the text will also be useful for the continuing
professional development of actuaries and other professionals in
insurance and related financial risk management industries.

\subsubsection*{Why is this good for the
profession?}\label{why-is-this-good-for-the-profession}
\addcontentsline{toc}{subsubsection}{Why is this good for the
profession?}

An online text is a type of open educational resource (OER). One
important benefit of an OER is that it equalizes access to knowledge,
thus permitting a broader community to learn about the actuarial
profession. Moreover, it has the capacity to engage viewers through
active learning that deepens the learning process, producing analysts
more capable of solid actuarial work. Why is this good for students and
teachers and others involved in the learning process?

Cost is often cited as an important factor for students and teachers in
textbook selection (see a recent post on the
\href{https://www.aei.org/publication/the-new-era-of-the-400-college-textbook-which-is-part-of-the-unsustainable-higher-education-bubble/}{\$400
textbook}). Students will also appreciate the ability to ``carry the
book around'' on their mobile devices.

\subsubsection*{Why loss data analytics?}\label{why-loss-data-analytics}
\addcontentsline{toc}{subsubsection}{Why loss data analytics?}

Although the intent is that this type of resource will eventually
permeate throughout the actuarial curriculum, one has to start
somewhere. Given the dramatic changes in the way that actuaries treat
data, loss data seems like a natural place to start. The idea behind the
name \emph{loss data analytics} is to integrate classical loss data
models from applied probability with modern analytic tools. In
particular, we seek to recognize that big data (including social media
and usage based insurance) are here and high speed computation s readily
available.

\subsubsection*{Project Goal}\label{project-goal}
\addcontentsline{toc}{subsubsection}{Project Goal}

The project goal is to have the actuarial community author our textbooks
in a collaborative fashion.

To get involved, please visit our
\href{https://sites.google.com/a/wisc.edu/loss-data-analytics/}{Loss
Data Analytics Project Site}.

\section*{Acknowledgements}\label{acknowledgements}
\addcontentsline{toc}{section}{Acknowledgements}

Edward Frees acknowledges the John and Anne Oros Distinguished Chair for
Inspired Learning in Business which provided seed money to support the
project. Frees and his Wisconsin colleagues also acknowledge a Society
of Actuaries Center of Excellence Grant that provided funding to support
work in dependence modeling and health initiatives.

We acknowledge the Society of Actuaries for permission to use problems
from their examinations.

We thank Rob Hyndman, Monash University, for allowing us to use his
excellent style files to produce the online version of the book.

We thank Yihui Xie and his colleagues at
\href{https://www.rstudio.com/}{Rstudio} for the
\href{https://bookdown.org/yihui/bookdown/}{R bookdown} package that
allows us to produce this book.

We also wish to acknowledge the support and sponsorship of the
\href{http://www.blackactuaries.org/}{International Association of Black
Actuaries} in our joint efforts to provide actuarial educational content
to all.

\includegraphics[width=0.25000\textwidth]{Figures/IABA.png}

\section*{Contributors}\label{contributors}
\addcontentsline{toc}{section}{Contributors}

The project goal is to have the actuarial community author our textbooks
in a collaborative fashion. The following contributors have taken a
leadership role in developing \emph{Loss Data Analytics}.

\begin{itemize}
\item
  \textbf{Zeinab Amin} is the Director of the Actuarial Science Program
  and Associate Dean for Undergraduate Studies of the School of Sciences
  and Engineering at the American University in Cairo (AUC). Amin holds
  a PhD in Statistics and is an Associate of the Society of Actuaries.
  Amin is the recipient of the 2016 Excellence in Academic Service Award
  and the 2009 Excellence in Teaching Award from AUC. Amin has designed
  and taught a variety of statistics and actuarial science courses.
  Amin's current area of research includes quantitative risk assessment,
  reliability assessment, general statistical modelling, and Bayesian
  statistics.
\item
  \textbf{Katrien Antonio}, KU Leuven
\item
  \textbf{Jan Beirlant}, KU Leuven
\item
  \textbf{Carolina Castro} - University of Buenos Aires
\end{itemize}

\begin{itemize}
\tightlist
\item
  \textbf{Curtis Gary Dean} is the Lincoln Financial Distinguished
  Professor of Actuarial Science at Ball State University. He is a
  Fellow of the Casualty Actuarial Society and a CFA charterholder. He
  has extensive practical experience as an actuary at American States
  Insurance, SAFECO, and Travelers. He has served the CAS and actuarial
  profession as chair of the Examination Committee, first
  editor-in-chief for \emph{Variance: Advancing the Science of Risk},
  and as a member of the Board of Directors and the Executive Council.
  He contributed a chapter to \emph{Predictive Modeling Applications in
  Actuarial Science} published by Cambridge University Press.
\end{itemize}

\begin{itemize}
\tightlist
\item
  \textbf{Edward W. (Jed) Frees} is an emeritus professor, formerly the
  Hickman-Larson Chair of Actuarial Science at the University of
  Wisconsin-Madison. He is a Fellow of both the Society of Actuaries and
  the American Statistical Association. He has published extensively (a
  four-time winner of the Halmstad and Prize for best paper published in
  the actuarial literature) and has written three books. He also is a
  co-editor of the two-volume series \emph{Predictive Modeling
  Applications in Actuarial Science} published by Cambridge University
  Press.
\end{itemize}

\begin{itemize}
\item
  \textbf{Guojun Gan} is an assistant professor in the Department of
  Mathematics at the University of Connecticut, where he has been since
  August 2014. Prior to that, he worked at a large life insurance
  company in Toronto, Canada for six years. He received a BS degree from
  Jilin University, Changchun, China, in 2001 and MS and PhD degrees
  from York University, Toronto, Canada, in 2003 and 2007, respectively.
  His research interests include data mining and actuarial science. He
  has published several books and papers on a variety of topics,
  including data clustering, variable annuity, mathematical finance,
  applied statistics, and VBA programming.
\item
  \textbf{Lisa Gao} is a doctoral student at the University of
  Wisconsin-Madison.
\item
  \textbf{José Garrido}, Concordia University
\end{itemize}

\begin{itemize}
\tightlist
\item
  \textbf{Noriszura Ismail} is a Professor and Head of Actuarial Science
  Program, Universiti Kebangsaan Malaysia (UKM). She specializes in Risk
  Modelling and Applied Statistics. She obtained her BSc and MSc
  (Actuarial Science) in 1991 and 1993 from University of Iowa, and her
  PhD (Statistics) in 2007 from UKM. She also passed several papers from
  Society of Actuaries in 1994. She has received several research grants
  from Ministry of Higher Education Malaysia (MOHE) and UKM, totaling
  about MYR1.8 million. She has successfully supervised and
  co-supervised several PhD students (13 completed and 11 on-going). She
  currently has about 180 publications, consisting of 88 journals and 95
  proceedings.
\end{itemize}

\begin{itemize}
\item
  \textbf{Joseph H.T. Kim}, Ph.D., FSA, CERA, is Associate Professor of
  Applied Statistics at Yonsei University, Seoul, Korea. He holds a
  Ph.D.~degree in Actuarial Science from the University of Waterloo, at
  which he taught as Assistant Professor. He also worked in the life
  insurance industry. He has published papers in \emph{Insurance
  Mathematics and Economics}, \emph{Journal of Risk and Insurance},
  \emph{Journal of Banking and Finance}, \emph{ASTIN Bulletin}, and
  \emph{North American Actuarial Journal}, among others.
\item
  \textbf{Shyamalkumar Nariankadu} - University of Iowa
\end{itemize}

\begin{itemize}
\item
  \textbf{Nii-Armah Okine} is a dissertator at the business school of
  University of Wisconsin-Madison with a major in actuarial science. He
  obtained his master's degree in Actuarial science from Illinois State
  University. His research interests includes micro-level reserving,
  joint longitudinal-survival modeling, dependence modelling, micro
  insurance and machine learning.
\item
  \textbf{Margie Rosenberg} - University of Wisconsin
\end{itemize}

\begin{itemize}
\item
  \textbf{Emine Selin Sarıdaş} is a doctoral candidate in the Statistics
  department of Mimar Sinan University. She holds a bachelor degree in
  Actuarial Science with a minor in Economics and a master degree in
  Actuarial Science from Hacettepe University. Her research interest
  includes dependence modeling, regression, loss models and life
  contingencies.
\item
  \textbf{Peng Shi} - University of Wisconsin - Madison
\item
  \textbf{Jianxi Su}, Purdue University
\item
  \textbf{Tim Verdonck}, KU Leuven
\end{itemize}

\begin{itemize}
\tightlist
\item
  \textbf{Krupa Viswanathan} is an Associate Professor in the Risk,
  Insurance and Healthcare Management Department in the Fox School of
  Business, Temple University. She is an Associate of the Society of
  Actuaries. She teaches courses in Actuarial Science and Risk
  Management at the undergraduate and graduate levels. Her research
  interests include corporate governance of insurance companies, capital
  management, and sentiment analysis. She received her Ph.D.~from The
  Wharton School of the University of Pennsylvania.
\end{itemize}

\section*{Reviewers}\label{reviewers}
\addcontentsline{toc}{section}{Reviewers}

The project goal is to have the actuarial community author our textbooks
in a collaborative fashion. Part of the writing process involves many
reviewers who generously donated their time to help make this book
better. They are:

\begin{itemize}
\tightlist
\item
  Chunsheng Ban, Ohio State University
\item
  Vytaras Brazauskas, University of Wisconsin - Milwaukee
\item
  Chun Yong Chew, Universiti Tunku Abdul Rahman (UTAR)
\item
  Eren Dodd, University of Southampton
\item
  Gordon Enderle, University of Wisconsin - Madison
\item
  Rob Erhardt, Wake Forest University
\item
  Liang (Jason) Hong, Robert Morris University
\item
  Hirokazu (Iwahiro) Iwasawa
\item
  Himchan Jeong, University of Connecticut
\item
  Paul Herbert Johnson, University of Wisconsin - Madison
\item
  Samuel Kolins, Lebonan Valley College
\item
  Andrew Soon-Yong Kwon, Zurich Re
\item
  Ambrose Lo, University of Iowa
\item
  Mark Maxwell, University of Texas at Austin
\item
  Tatjana Miljkovic, Miami University
\item
  Bell Ouelega, American University in Cairo
\item
  Zhiyu (Frank) Quan, University of Connecticut
\item
  Jiandong Ren, Western University
\item
  Rajesh V. Sahasrabuddhe, Oliver Wyman
\item
  Ranee Thiagarajah, Illinois State University
\item
  Ping Wang, Saint Johns University
\item
  Chengguo Weng, University of Waterloo
\item
  Toby White, Drake University
\item
  Michelle Xia, Northern Illinois University
\item
  Di (Cindy) Xu, University of Nebraska - Lincoln
\item
  Lina Xu, Columbia University
\item
  Jorge Yslas, University of Copenhagen
\item
  Jeffrey Zheng, Temple University
\item
  Hongjuan Zhou, Arizona State University
\end{itemize}

\chapter{Main Time Line}\label{main-time-line}

Placeholder

\section{Relevance of Analytics}\label{S:Intro}

\subsection{What is Analytics?}\label{what-is-analytics}

\subsection{Short and Long-term
Insurance}\label{short-and-long-term-insurance}

\subsection{Insurance Processes}\label{S:InsProcesses}

\section{Insurance Company Operations}\label{S:PredModApps}

\subsection{Initiating Insurance}\label{initiating-insurance}

\subsection{Renewing Insurance}\label{renewing-insurance}

\subsection{Claims and Product
Management}\label{claims-and-product-management}

\subsection{Loss Reserving}\label{S:Reserving}

\section{Case Study: Wisconsin Property Fund}\label{S:LGPIF}

\subsection{Fund Claims Variables: Frequency and
Severity}\label{S:OutComes}

\subsection{Fund Rating Variables}\label{S:FundVariables}

\subsection{Fund Operations}\label{fund-operations}

\section{Further Resources and
Contributors}\label{Intro-further-reading-and-resources}

\chapter{Frequency Modeling}\label{C:Frequency-Modeling}

Placeholder

\section{Frequency Distributions}\label{S:frequency-distributions}

\subsection{How Frequency Augments Severity
Information}\label{S:how-frequency-augments-severity-information}

\subsubsection{Basic Terminology}\label{S:basic-terminology}

\subsubsection{The Importance of
Frequency}\label{S:the-importance-of-frequency}

\subsubsection{Why Examine Frequency
Information}\label{S:why-examine-frequency-information}

\section{Basic Frequency
Distributions}\label{S:basic-frequency-distributions}

\subsection{Foundations}\label{S:foundations}

\subsection{Moment and Probability Generating
Functions}\label{S:generating-functions}

\subsection{Important Frequency
Distributions}\label{S:important-frequency-distributions}

\subsubsection{Binomial Distribution}\label{S:binomial-distribution}

\subsubsection{Poisson Distribution}\label{S:poisson-distribution}

\subsubsection{Negative Binomial
Distribution}\label{S:negative-binomial-distribution}

\section{The (a, b, 0) Class}\label{S:the-a-b-0-class}

\section{Estimating Frequency
Distributions}\label{S:estimating-frequency-distributions}

\subsection{Parameter estimation}\label{S:parameter-estimation}

\subsection{Frequency Distributions
MLE}\label{S:frequency-distributions-mle}

\section{Other Frequency
Distributions}\label{S:other-frequency-distributions}

\subsection{Zero Truncation or
Modification}\label{S:zero-truncation-or-modification}

\section{Mixture Distributions}\label{S:mixture-distributions}

\section{Goodness of Fit}\label{S:goodness-of-fit}

\section{Exercises}\label{S:exercises}

\section{Quiz}\label{S:quiz}

\section{R Code for Plots in this
Chapter}\label{r-code-for-plots-in-this-chapter}

\section{Further Resources and
Contributors}\label{Freq-further-reading-and-resources}

\subsubsection*{Contributors}\label{contributors-1}
\addcontentsline{toc}{subsubsection}{Contributors}

\chapter{Varying Scale Gamma
Densities}\label{varying-scale-gamma-densities}

Placeholder

\section{Further Resources and
Contributors}\label{LM-further-reading-and-resources}

\chapter{Model Selection and Estimation}\label{C:ModelSelection}

Placeholder

\section{Nonparametric Inference}\label{S:MS:NonParInf}

\subsection{Nonparametric Estimation}\label{nonparametric-estimation}

\subsubsection{Moment Estimators}\label{S:MS:MomentEstimator}

\subsubsection{Empirical Distribution
Function}\label{empirical-distribution-function}

\subsubsection{Quantiles}\label{S:MS:QuantileEstimator}

\subsubsection{Density Estimators}\label{density-estimators}

\subsection{Tools for Model Selection}\label{S:MS:ToolsModelSelection}

\subsubsection{Graphical Comparison of
Distributions}\label{graphical-comparison-of-distributions}

\subsubsection{Statistical Comparison of
Distributions}\label{S:MS:Tools:Stats}

\subsection{Starting Values}\label{starting-values}

\subsubsection{Method of Moments}\label{method-of-moments}

\subsubsection{Percentile Matching}\label{percentile-matching}

\section{Model Selection}\label{S:MS:ModelSelection}

\subsection{Iterative Model Selection}\label{iterative-model-selection}

\subsection{Model Selection Based on a Training
Dataset}\label{model-selection-based-on-a-training-dataset}

\subsection{Model Selection Based on a Test
Dataset}\label{model-selection-based-on-a-test-dataset}

\subsection{Model Selection Based on
Cross-Validation}\label{model-selection-based-on-cross-validation}

\section{Estimation using Modified Data}\label{S:MS:ModifiedData}

\subsection{Parametric Estimation using Modified
Data}\label{parametric-estimation-using-modified-data}

\subsubsection{Parametric Estimation using Grouped
Data}\label{S:MS:GroupedData}

\subsubsection{Censored Data}\label{censored-data}

\subsubsection{Truncated Data}\label{truncated-data}

\subsubsection{Parametric Estimation using Censored and Truncated
Data}\label{parametric-estimation-using-censored-and-truncated-data}

\subsection{Nonparametric Estimation using Modified
Data}\label{nonparametric-estimation-using-modified-data}

\subsubsection{Grouped Data}\label{grouped-data}

\subsubsection{Right-Censored Empirical Distribution
Function}\label{right-censored-empirical-distribution-function}

\subsubsection{Right-Censored, Left-Truncated Empirical Distribution
Function}\label{right-censored-left-truncated-empirical-distribution-function}

\section{Bayesian Inference}\label{S:MS:BayesInference}

\subsection{Bayesian Model}\label{bayesian-model}

\subsection{Decision Analysis}\label{decision-analysis}

\subsection{Posterior Distribution}\label{posterior-distribution}

\section{Further Resources and
Contributors}\label{MS:further-reading-and-resources}

\subsubsection*{Exercises}\label{exercises}
\addcontentsline{toc}{subsubsection}{Exercises}

\chapter{Lorenz Curve}\label{lorenz-curve}

Placeholder

\section*{Technical Supplement A. Gini
Statistic}\label{technical-supplement-a.-gini-statistic}
\addcontentsline{toc}{section}{Technical Supplement A. Gini Statistic}

\subsection*{TS A.1. The Classic Lorenz
Curve}\label{ts-a.1.-the-classic-lorenz-curve}
\addcontentsline{toc}{subsection}{TS A.1. The Classic Lorenz Curve}

\subsection*{TS A.2. Ordered Lorenz Curve and the Gini
Index}\label{ts-a.2.-ordered-lorenz-curve-and-the-gini-index}
\addcontentsline{toc}{subsection}{TS A.2. Ordered Lorenz Curve and the
Gini Index}

\subsubsection*{Ordered Lorenz Curve}\label{ordered-lorenz-curve}
\addcontentsline{toc}{subsubsection}{Ordered Lorenz Curve}

\subsubsection*{Gini Index}\label{gini-index}
\addcontentsline{toc}{subsubsection}{Gini Index}

\subsection*{TS A.3. Out-of-Sample
Validation}\label{ts-a.3.-out-of-sample-validation}
\addcontentsline{toc}{subsection}{TS A.3. Out-of-Sample Validation}

\subsubsection*{Discussion}\label{discussion}
\addcontentsline{toc}{subsubsection}{Discussion}

\chapter{Aggregate Loss Models}\label{C:AggLossModels}

Placeholder

\section{Introduction}\label{introduction}

\section{Individual Risk Model}\label{individual-risk-model}

\section{Collective Risk Model}\label{collective-risk-model}

\subsection{Moments and Distribution}\label{moments-and-distribution}

\subsection{Stop-loss Insurance}\label{stop-loss-insurance}

\subsection{Analytic Results}\label{analytic-results}

\subsection{Tweedie Distribution}\label{tweedie-distribution}

\section{Computing the Aggregate Claims
Distribution}\label{computing-the-aggregate-claims-distribution}

\subsection{Recursive Method}\label{recursive-method}

\subsection{Simulation}\label{simulation}

\section{Effects of Coverage
Modifications}\label{effects-of-coverage-modifications}

\subsection{Impact of Exposure on
Frequency}\label{impact-of-exposure-on-frequency}

\subsection{Impact of Deductibles on Claim
Frequency}\label{impact-of-deductibles-on-claim-frequency}

\subsection{Impact of Policy Modifications on Aggregate
Claims}\label{impact-of-policy-modifications-on-aggregate-claims}

\section{Further Resources and
Contributors}\label{AL-further-reading-and-resources}

\subsubsection*{Exercises}\label{exercises-1}
\addcontentsline{toc}{subsubsection}{Exercises}

\section{Generating Independent Uniform
Observations}\label{generating-independent-uniform-observations}

\section{Inverse Transform}\label{inverse-transform}

\section{How Many Simulated Values?}\label{how-many-simulated-values}

\chapter{Premium Calculation Fundamentals}\label{C:PremCalc}

This is a placeholder file

\chapter{Risk Classification}\label{C:RiskClass}

Placeholder

\section{Introduction}\label{S:RC:Introduction}

\section{Poisson Regression Model}\label{S:RC:PoissonRegression}

\subsection{Need for Poisson Regression}\label{S:RC:Need.Poi.reg}

\subsection{Poisson Regression}\label{poisson-regression}

\subsection{Incorporating Exposure}\label{incorporating-exposure}

\subsection{Exercises}\label{exercises-2}

\section{Categorical Variables and Multiplicative
Tariff}\label{S:CatVarMultiTarriff}

\subsection{Rating Ractors and Tariff}\label{rating-ractors-and-tariff}

\subsection{Multiplicative Tariff
Model}\label{multiplicative-tariff-model}

\subsection{Poisson Regression for Multiplicative
Tariff}\label{poisson-regression-for-multiplicative-tariff}

\subsection{Numerical Examples}\label{numerical-examples}

\section{Contributors and Further
Resources}\label{RC:further-reading-and-resources}

\subsubsection*{Further Reading and
References}\label{further-reading-and-references}
\addcontentsline{toc}{subsubsection}{Further Reading and References}

\subsubsection*{Contributor}\label{contributor}
\addcontentsline{toc}{subsubsection}{Contributor}

\section{Technical Supplement -- Estimating Poisson Regression
Models}\label{S:RC:mle-Pois-reg}

\chapter{Experience Rating Using Credibility
Theory}\label{C:Credibility}

Placeholder

\section{Introduction to Applications of Credibility
Theory}\label{introduction-to-applications-of-credibility-theory}

\section{Limited Fluctuation
Credibility}\label{limited-fluctuation-credibility}

\subsection{Full Credibility for Claim Frequency}\label{S:frequency}

\subsection{Full Credibility for Aggregate Losses and Pure
Premium}\label{full-credibility-for-aggregate-losses-and-pure-premium}

\subsection{Full Credibility for
Severity}\label{full-credibility-for-severity}

\subsection{Partial Credibility}\label{partial-credibility}

\section{Bühlmann Credibility}\label{buhlmann-credibility}

\subsection{\texorpdfstring{Credibility Z, \emph{EPV}, and
\emph{VHM}}{Credibility Z, EPV, and VHM}}\label{S:EPV-VHM-Z}

\section{Bühlmann-Straub Credibility}\label{buhlmann-straub-credibility}

\section{Bayesian Inference and
Bühlmann}\label{bayesian-inference-and-buhlmann}

\subsection{Gamma-Poisson Model}\label{gamma-poisson-model}

\subsection{Exact Credibility}\label{exact-credibility}

\section{Estimating Credibility
Parameters}\label{estimating-credibility-parameters}

\subsection{Full Credibility Standard for Limited Fluctuation
Credibility}\label{full-credibility-standard-for-limited-fluctuation-credibility}

\subsection{Nonparametric Estimation for Bühlmann and Bühlmann-Straub
Models}\label{nonparametric-estimation-for-buhlmann-and-buhlmann-straub-models}

\subsection{Semiparametric Estimation for Bühlmann and Bühlmann-Straub
Models}\label{semiparametric-estimation-for-buhlmann-and-buhlmann-straub-models}

\subsection{Balancing Credibility
Estimators}\label{balancing-credibility-estimators}

\section{Further Resources and
Contributors}\label{Cred-further-reading-and-resources}

\subsubsection*{Exercises}\label{exercises-3}
\addcontentsline{toc}{subsubsection}{Exercises}

\chapter{For the gamma distributions,
use}\label{for-the-gamma-distributions-use}

Placeholder

\subsection{Classification Based on
Moments}\label{classification-based-on-moments}

\subsection{Comparison Based on Limiting Tail
Behavior}\label{comparison-based-on-limiting-tail-behavior}

\section{Risk Measures}\label{S:RiskMeasure}

\subsection{Value-at-Risk}\label{value-at-risk}

\subsection{Tail Value-at-Risk}\label{tail-value-at-risk}

\subsection{Properties of risk
measures}\label{properties-of-risk-measures}

\subsection{Proportional Reinsurance}\label{S:ProportionalRe}

\subsubsection{Quota Share is Desirable for
Reinsurers}\label{quota-share-is-desirable-for-reinsurers}

\subsubsection{Optimizing Quota Share Agreements for
Insurers}\label{optimizing-quota-share-agreements-for-insurers}

\subsection{Non-Proportional Reinsurance}\label{S:NonProportionalRe}

\subsubsection{Excess of Loss}\label{excess-of-loss}

\subsection{Additional Reinsurance Treaties}\label{S:AdditionalRe}

\subsubsection{Surplus Share Proportional
Treaty}\label{surplus-share-proportional-treaty}

\subsubsection{Layers of Coverage}\label{layers-of-coverage}

\chapter{Loss Reserving}\label{C:LossReserves}

This is a placeholder file

\chapter{Experience Rating using Bonus-Malus}\label{C:BonusMalus}

This is a placeholder file

\textbf{Bonus-Malus}

Bonus-malus system, which is used interchangeably as ``no-fault
discount'', ``merit rating'', ``experience rating'' or ``no-claim
discount'' in different countries, is based on penalizing insureds who
are responsible for one or more claims by a premium surcharge, and
awarding insureds with a premium discount if they do not have any claims
(Frangos and Vrontos, 2001). Insurers use bonus-malus systems for two
main purposes; firstly, to encourage drivers to drive more carefully in
a year without any claims, and secondly, to ensure insureds to pay
premiums proportional to their risks which are based on their claims
experience.

\textbf{NCD and Experience Rating}

No Claim Discount (NCD) system is an experience rating system commonly
used in motor insurance. NCD system represents an attempt to categorize
insureds into homogeneous groups who pay premiums based on their claims
experience. Depending on the rules in the scheme, new policyholders may
be required to pay full premium initially, and obtain discounts in the
future years as a results of claim-free years.

\textbf{Hunger for Bonus }

An NCD system rewards policyholders for not making any claims during a
year, or in other words, it grants a bonus to a careful driver. This
bonus principle may affect policy holders' decisions whether to claim or
not to claim, especially when involving accidents with slight damages,
which is known as `hunger for bonus' phenomenon (Philipson, 1960). The
option of `hunger for bonus' implemented on insureds under an NCD system
may reduce insurers' claim costs, and may be able to offset the expected
decrease in premium income.

\chapter{Data Systems}\label{C:DataSystems}

Placeholder

\section{Data}\label{data}

\subsection{Data Types and Sources}\label{data-types-and-sources}

\subsection{Data Structures and
Storage}\label{data-structures-and-storage}

\subsection{Data Quality}\label{data-quality}

\subsection{Data Cleaning}\label{data-cleaning}

\section{Data Analysis Preliminary}\label{data-analysis-preliminary}

\subsection{Data Analysis Process}\label{S:process}

\subsection{Exploratory versus
Confirmatory}\label{exploratory-versus-confirmatory}

\subsection{Supervised versus
Unsupervised}\label{supervised-versus-unsupervised}

\subsection{Parametric versus
Nonparametric}\label{parametric-versus-nonparametric}

\subsection{Explanation versus Prediction}\label{S:expred}

\subsection{Data Modeling versus Algorithmic
Modeling}\label{data-modeling-versus-algorithmic-modeling}

\subsection{Big Data Analysis}\label{big-data-analysis}

\subsection{Reproducible Analysis}\label{reproducible-analysis}

\subsection{Ethical Issues}\label{ethical-issues}

\section{Data Analysis Techniques}\label{data-analysis-techniques}

\subsection{Exploratory Techniques}\label{exploratory-techniques}

\subsection{Descriptive Statistics}\label{descriptive-statistics}

\subsubsection{Principal Component
Analysis}\label{principal-component-analysis}

\subsection{Cluster Analysis}\label{cluster-analysis}

\subsection{Confirmatory Techniques}\label{confirmatory-techniques}

\subsubsection{Linear Models}\label{linear-models}

\subsubsection{Generalized Linear
Models}\label{generalized-linear-models}

\subsubsection{Tree-based Models}\label{tree-based-models}

\section{Some R Functions}\label{some-r-functions}

\section{Summary}\label{summary}

\section{Further Resources and
Contributors}\label{DS:further-reading-and-resources}

\chapter{Dependence Modeling}\label{C:DependenceModel}

Placeholder

\section{Variable Types}\label{S:VarTypes}

\subsection{Qualitative Variables}\label{S:QuaVar}

\subsection{Quantitative Variables}\label{S:QuanVar}

\subsection{Multivariate Variables}\label{multivariate-variables}

\section{Classic Measures of Scalar Associations}\label{S:Measures}

\subsection{Association Measures for Quantitative
Variables}\label{association-measures-for-quantitative-variables}

\subsubsection{Pearson Correlation}\label{pearson-correlation}

\subsection{Pearson correlation between Claim and
Coverage}\label{pearson-correlation-between-claim-and-coverage}

\subsection{Pearson correlation between Claim and
log(Coverage)}\label{pearson-correlation-between-claim-and-logcoverage}

\subsection{Rank Based Measures}\label{rank-based-measures}

\subsubsection{Spearman's Rho}\label{spearmans-rho}

\subsection{Spearman correlation between Claim and
Coverage}\label{spearman-correlation-between-claim-and-coverage}

\subsection{Spearman correlation between Claim and
log(Coverage)}\label{spearman-correlation-between-claim-and-logcoverage}

\subsubsection{Kendall's Tau}\label{kendalls-tau}

\subsection{Kendall's tau correlation between Claim and
Coverage}\label{kendalls-tau-correlation-between-claim-and-coverage}

\subsection{Kendall's tau correlation between Claim and
log(Coverage)}\label{kendalls-tau-correlation-between-claim-and-logcoverage}

\subsection{Nominal Variables}\label{nominal-variables}

\subsubsection{Bernoulli Variables}\label{bernoulli-variables}

\subsubsection{Categorical Variables}\label{categorical-variables}

\subsubsection{Ordinal Variables}\label{ordinal-variables}

\subsubsection{Parametric Approach Using Normal Based
Correlations}\label{parametric-approach-using-normal-based-correlations}

\subsubsection{Interval Variables}\label{interval-variables}

\subsubsection{Discrete and Continuous
Variables}\label{discrete-and-continuous-variables}

\section{Introduction to Copulas}\label{S:Copula}

\section{Application Using Copulas}\label{S:CopAppl}

\subsection{Data Description}\label{data-description}

\subsection{Marginal Models}\label{marginal-models}

\subsection{Probability Integral
Transformation}\label{probability-integral-transformation}

\subsection{Joint Modeling with Copula
Function}\label{joint-modeling-with-copula-function}

\section{Types of Copulas}\label{S:CopTyp}

\subsection{Elliptical Copulas}\label{elliptical-copulas}

\subsection{Archimedian Copulas}\label{archimedian-copulas}

\subsubsection{Clayton Copula}\label{clayton-copula}

\subsubsection{Gumbel-Hougaard copula}\label{gumbel-hougaard-copula}

\subsection{Properties of Copulas}\label{properties-of-copulas}

\subsubsection{Bounds on Association}\label{bounds-on-association}

\subsubsection{Measures of Association}\label{measures-of-association}

\subsubsection{Tail Dependency}\label{tail-dependency}

\section{Why is Dependence Modeling Important?}\label{S:CopImp}

\subsection{Normal Copula}\label{normal-copula}

\subsection{Normal Copula}\label{normal-copula-1}

\section{Further Resources and
Contributors}\label{Dep:further-reading-and-resources}

\section*{Technical Supplement A. Other Classic Measures of Scalar
Associations}\label{technical-supplement-a.-other-classic-measures-of-scalar-associations}
\addcontentsline{toc}{section}{Technical Supplement A. Other Classic
Measures of Scalar Associations}

\subsection*{A.1. Blomqvist's Beta}\label{a.1.-blomqvists-beta}
\addcontentsline{toc}{subsection}{A.1. Blomqvist's Beta}

\subsection{Blomqvist's beta correlation between Claim and
Coverage}\label{blomqvists-beta-correlation-between-claim-and-coverage}

\subsection{Blomqvist's beta correlation between Claim and
log(Coverage)}\label{blomqvists-beta-correlation-between-claim-and-logcoverage}

\subsection{Blomqvist's beta correlation between Claim and
Coverage}\label{blomqvists-beta-correlation-between-claim-and-coverage-1}

\subsection{Blomqvist's beta correlation between Claim and
log(Coverage)}\label{blomqvists-beta-correlation-between-claim-and-logcoverage-1}

\subsection*{A.2. Nonparametric Approach Using Spearman Correlation with
Tied
Ranks}\label{a.2.-nonparametric-approach-using-spearman-correlation-with-tied-ranks}
\addcontentsline{toc}{subsection}{A.2. Nonparametric Approach Using
Spearman Correlation with Tied Ranks}

\chapter{Appendix A: Review of Statistical Inference}\label{C:AppA}

Placeholder

\section{Basic Concepts}\label{S:AppA:BASIC}

\subsection{Random Sampling}\label{random-sampling}

\subsection{Sampling Distribution}\label{sampling-distribution}

\subsection{Central Limit Theorem}\label{central-limit-theorem}

\section{Point Estimation and Properties}\label{S:AppA:PE}

\subsection{Method of Moments
Estimation}\label{method-of-moments-estimation}

\subsection{Maximum Likelihood
Estimation}\label{maximum-likelihood-estimation}

\section{Interval Estimation}\label{S:AppA:IE}

\subsection{Exact Distribution for Normal Sample
Mean}\label{S:AppA:IE:ED}

\subsection{Large-sample Properties of
MLE}\label{large-sample-properties-of-mle}

\subsection{Confidence Interval}\label{confidence-interval}

\section{Hypothesis Testing}\label{S:AppA:HT}

\subsection{Basic Concepts}\label{basic-concepts}

\subsection{\texorpdfstring{Student-\(t\) test based on
MLE}{Student-t test based on MLE}}\label{student-t-test-based-on-mle}

\subsection{Likelihood Ratio Test}\label{S:AppA:HT:LRT}

\subsection{Information Criteria}\label{S:AppA:HT:IC}

\chapter{Appendix B: Iterated Expectations}\label{C:AppB}

Placeholder

\section{Conditional Distribution and Conditional
Expectation}\label{S:AppB:CD}

\subsection{Conditional Distribution}\label{conditional-distribution}

\subsubsection{Discrete Case}\label{discrete-case}

\subsubsection{Continuous Case}\label{continuous-case}

\subsection{Conditional Expectation and Conditional
Variance}\label{conditional-expectation-and-conditional-variance}

\subsubsection{Discrete Case}\label{discrete-case-1}

\subsubsection{Continuous Case}\label{continuous-case-1}

\section{Iterated Expectations and Total Variance}\label{S:AppB:IE}

\subsection{Law of Iterated
Expectations}\label{law-of-iterated-expectations}

\subsection{Law of Total Variance}\label{law-of-total-variance}

\subsection{Application}\label{application}

\chapter{Appendix C: Maximum Likelihood Theory}\label{C:AppC}

Placeholder

\section{Likelihood Function}\label{S:AppC:LF}

\subsection{Likelihood and Log-likelihood
Functions}\label{likelihood-and-log-likelihood-functions}

\subsection{Properties of Likelihood
Functions}\label{properties-of-likelihood-functions}

\section{Maximum Likelihood Estimators}\label{S:AppC:MLE}

\subsection{Definition and Derivation of
MLE}\label{definition-and-derivation-of-mle}

\subsection{Asymptotic Properties of
MLE}\label{asymptotic-properties-of-mle}

\subsection{Use of Maximum Likelihood
Estimation}\label{use-of-maximum-likelihood-estimation}

\section{Statistical Inference Based on Maximum Likelhood
Estimation}\label{S:AppC:SI}

\subsection{Hypothesis Testing}\label{hypothesis-testing}

\subsection{MLE and Model Validation}\label{S:AppC:MLEModelVal}

\bibliography{Bibliography/LDAReferenceB.bib}


\end{document}
